%
% File Name: supplement.tex
% Author:    Aditya Ramesh
% Date:      01/02/2015
% Contact:   _@adityaramesh.com
%

\documentclass[11pt,a4paper]{article}

% Bibliography management.
\usepackage{hyperref}

% Font configuration.
\usepackage[
	activate={true,nocompatibility},
	tracking=true
]{microtype}
\usepackage{fontspec}
\usepackage{mathpazo}
\setmainfont[Ligatures = TeX]{TeX Gyre Pagella}
\setsansfont[Ligatures = TeX, Scale = 0.9]{Source Code Pro}
\setmonofont[Ligatures = TeX, Scale = 0.9]{Source Code Pro}

\title{Supplement to Report}
\author{Aditya Ramesh}
\date{}

\begin{document}

\maketitle

Several libraries (written by me) were used to run the experiments mentioned in
the report. These libraries are described below.
\begin{description}
\item[\textsf{ccbase}] A C++ utility library for formatting output, detecting
platform-specific features, unit testing, error handling, loading dynamic
libraries, and working with the filesystem. Downloadable at
\url{https://github.com/adityaramesh/ccbase}.

\item[\textsf{ctop}] A C++ library for analyzing system topology information.
Downloadable at \url{https://github.com/adityaramesh/ctop}.

\item[\textsf{gcd}] A small C++14 library used to solve systems of linear
congruences. Downloadable at \url{https://github.com/adityaramesh/gcd}.

\item[\textsf{z1d}] A C++14 library for one-dimensional root-finding.
Downloadable at \url{https://github.com/adityaramesh/z1d}.

\item[\textsf{neo}] A C++ library for high-performance binary IO. Included in
archive.

\item[\textsf{ntl}] A C++ library for training feed-forward neural networks.
Included in archive. The \textsf{cuda-convnet} library, written by Alex
Krizhevsky, was used to help validate correctness during the development of
\textsf{ntl}.

\item[\textsf{ndmath}] A C++ library for working with $n$-dimensional arrays.
Included in archive.
\end{description}

\end{document}
